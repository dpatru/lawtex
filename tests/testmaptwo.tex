\providecommand{\documentclassflag}{}
\documentclass[12pt,\documentclassflag]{lawbrief} 

\usepackage[margin=1in]{geometry}
\usepackage{newcent,microtype}
\usepackage{setspace,xcolor}
\usepackage[hyperindex=false,linkbordercolor=white,pdfborder={0 0 0}]{hyperref}
\usepackage{test}
\usepackage{showexpl} % LTXexample blocks
\usepackage{xstring}
\usepackage{listings}
\begin{document}

\section{Testing maptwo}

We test a looping macro that applies a two-argument command to each
pair of two lists. Macro \lstinline|\maptwo| is called with five
parameters: the input list separator, the output list separator, the
macro to apply to each pair, and the two lists.

\begin{LTXexample}
  \long\def\maptwo#1#2#3#4#5{{% create a block to restrict the scope
    \def\endlist{\endlist}% mark the end of the list
    \makeatletter
    \long\def\@maptwo##1##2#1##3\endlist##4#1##5\endlist{%
      \ifx\empty##2\empty\relax\else\ifx\empty##4\empty\relax\else{}##1{##2}{##4}\fi\fi%
      \ifx\empty##3\empty\relax\else\ifx\empty##5\empty\relax\else%
      #2\@maptwo{##1}##3\endlist{}##5\endlist\fi\fi}
    \@maptwo{#3}#4#1\endlist{}#5#1\endlist}}
  \def\hello#1#2{hello #1 #2}

  Testing \textbackslash{}maptwo:

  Call \lstinline|\maptwo| with input list separator semicolon,
  output list separator \lstinline|\par\vspace{5mm}|, macro \lstinline|\hello|, and lists
  \lstinline|one;two; three;four| and \lstinline|a;b;c|.
  
  \maptwo;{\par\vspace{5mm}}{\hello}{one;two; three;four}{a;b;c}
  
\end{LTXexample}

\section{Defining a macro in a macro}

Basic: 

\begin{LTXexample}
  \def\mklst#1{\def#1{one,two,three}}
  \mklst{\hello}
  hello is now: 
  \hello 
\end{LTXexample}

If we want to use a macro in its own definition, we need to edef it,  
otherwise \TeX\ gets confused and goes into an infinite loop.  

 \begin{LTXexample}
  \def\mklst#1{\edef#1{#1,one,two,three}}
  \def\hello{zero}
  \mklst{\hello}
  hello is now:   \hello 
\end{LTXexample}

\begin{LTXexample}
  \def\addtolst#1#2,{\ifx\relax#2\relax\def\next{\relax}%
    List is #1\par\else Defining #1\par\edef#1{#1;#2}%
  \def\next{\addtolst{#1}}\fi\next}

\def\mylst{zero}

\addtolst{\mylst}one,two,three,,
\end{LTXexample}

\section{Undoing maptwo}

We can undo \textbackslash{}maptwo by taking a list and decomposing it
into two lists.

\begin{LTXexample}
  \def\undoMapTwo#1#2#3 #4,{% Expects a comma separated list with
                            % ending flag being two commas separated
                            % by a space.
    \def\next{\undoMapTwo{#1}{#2}}%
    \ifx\relax#3\relax Lists are: (#1) and (#2)\par\def\next{\relax}% Stop
    \else\ifx\empty#1\empty Initializing the macros\par\edef#1{#3}\edef#2{#4}% first time
    \else Adding to the list #1 \edef#1{#1, #3} Adding to the list #2\par \edef#2{#2, #4}%
    \par\fi\fi\next}

  \def\one{}\def\two{} % TeX complains if these are not defined.
  \undoMapTwo\one\two 1 one,2 two,3 three, ,

  List one is \one.

  List two is \two.
  
\end{LTXexample}  
\end{document}

