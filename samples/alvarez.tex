\providecommand{\documentclassflag}{}
\documentclass[12pt,\documentclassflag]{lawbrief} 

\usepackage[margin=1in]{geometry}
\usepackage{newcent,microtype}
\usepackage{setspace,xcolor}
\usepackage[hyperindex=false,linkbordercolor=white,pdfborder={0 0 0}]{hyperref}

%%Cases
\makeandletter
\citecase{Ashcroft v. American Civil Liberties Union, 535 U.S. 564 (2002)}
\citecase[Playboy]{United States v. Playboy Entertainment Group, Inc., 529 U.S. 803 (2000)}
\citecase{Chaplinsky v. New Hampshire, 315 U.S. 568 (1942)}
\citecase{R.A.V. v. City of St. Paul, Minnesota, 505 U.S. 377 (1992)}
\citecase{United States v. Stevens, 130 S. Ct. 1577 (2010)}
\citecase[Carrigan]{Nevada Com'n on Ethics v. Carrigan, 131 S. Ct. 2343 (2011)}
\citecase[Lorillard]{Lorillard Tobacco Co. v. Reilly, 533 U.S. 525 (2001)}
\citecase[Ferber]{New York v. Ferber, 458 U.S. 747 (1982)}
\citecase{Brown v. Entertainment Merchants Ass'n, 131 S. Ct. 2729 (2011)}
\citecase{Winters v. New York, 333 U.S. 507 (1948)}
\citecase[Hustler]{Hustler Magazine, Inc. v. Falwell, 485 U.S. 46 (1988)}
\citecase{Gertz v. Robert Welch, Inc., 418 U.S. 323 (1974)}
\citecase[BE & K Construction]{BE & K Construction Co. v. N.L.R.B., 536 U.S. 516 (2002)}
\citecase{Keeton v. Hustler Magazine, Inc., 465 U.S. 770 (1984)}
\citecase[Bill Johnson's]{Bill Johnson's Restaurants, Inc. v. N.L.R.B., 461 U.S. 731 (1983)}
\citecase{Herbert v. Lando, 441 U.S. 153 (1979)}
\citecase{Garrison v. Louisiana, 379 U.S. 64 (1964)}

\newcase{Alvarez I}{United States v. Alvarez \emph{(Alvarez I)}}{Docket 2:07-cr-01035, No. 29}{}{(C.D. Cal., Apr. 9, 2008)}
\newcase{Alvarez II}{United States v. Alvarez \emph{(Alvarez II)}}{617 F.3d}{1198}{(9th Cir. 2010), \noexpand\emph{reh'g en banc denied}, 638 F.3d 666 (9th Cir. 2011)}
\SetIndexName{Alvarez II}{United States v. Alvarez \emph{(Alvarez II)}, 617 F.3d 1198 (9th Cir. 2010)}

\citecase{Snyder v. Phelps, 131 S. Ct. 1207 (2011)}
\citecase[New York Times]{New York Times Co. v. Sullivan, 376 U.S. 254 (1964)}
\citecase{Va. State Bd. of Pharmacy v. Va. Citizens Consumer Council, Inc., 425 U.S. 748 (1976)}
\citecase[Johnson]{Texas v. Johnson, 491 U.S. 397 (1989)}
\citecase[Black]{Virginia v. Black, 538 U.S. 343 (2003)}
\newcase{Madigan}{Illinois, \emph{ex rel.} Madigan v. Telemarketing Associates, Inc.}{538 U.S.}{600}{(2003)}
\citecase{United States v. Williams, 553 U.S. 285 (2008)}
\citecase{Meyer v. Grant, 486 U.S. 414 (1988)}
\citecase{Carpenter v. United States, 484 U.S. 19 (1987)}
\citecase{NAACP v. Button, 371 U.S. 415 (1963)}
\citecase{Riley v. National Federation of the Blind of North Carolina, Inc., 487 U.S. 781 (1988)}
\citecase{Schaumburg v. Citizens for a Better Environment, 444 U.S. 620 (1980)}
\citecase[Munson]{Secretary of State of Md. v. Joseph H. Munson Co., 467 U.S. 947 (1984)}
\citecase[Dun & Bradstreet]{Dun & Bradstreet, Inc. v. Greenmoss Builders, Inc., 472 U.S. 749 (1985)}
\citecase{United States v. Dunnigan, 507 U.S. 87 (1993)}
\makeandtab

%The following line add ``U.S. Const. Amend. I'' to the Statutes index section, 
% uses the aa@ prefix to order it first in the list, 
% and marks it passim
\index{Statute}{aa@\textsc{U.S. Const.} amend. I|idxpassim}
\index{Statute}{ab@Stolen Valor Act, 18 U.S.C. \S 704|idxpassim}

\newmisc{151 Cong. Rec.}{151 Cong. Rec. S12684-01\pin{, }{}}
\newmisc{152 Cong. Rec.}{152 Cong. Rec. H8819-01\pin{, }{}}
\newmisc{Restatement of Torts}{Restatement of Torts\pin{ }{} (1938)}

\begin{document}

\docket{No. 11-210} 
\titlegraphic{SupremeCourt.pdf}
\petitioner{United States of America}
\respondent{Xavier Alvarez}
\circuit{Ninth}
\brieffor{Respondent}
\author{Anonymous \# 177\\{\it Counsel for Respondent}}
\address{127 Wall Street \\ New Haven, CT 06510\\ (203) 555-1234}

\questionpresented{{ The Stolen Valor Act criminalizes pure speech based on that speech's content, namely: false claims that one has been awarded a military medal.  The Act does not require proof of harm, intent to defraud, or that anyone was deceived by such speech---only that the speech was false. Does Congress thus have the power to ban any intentionally false statements of fact, solely because they are false? \doublespacing\par }}

%This commands creates the title page, table of contents, and table of authorities
\makeandletter
\makefrontmatter
\makeandtab

%Sets the formatting for the entire document after the front matter
\parindent=2.5em 
\setlength{\parskip}{1.25ex plus 2ex minus .5ex} 
\setstretch{1.40}  

\section{Opinions Below} 

The opinion of the District Court for the Central District of California is unreported. \citetext{\Reporter{Alvarez I} \Parenthetical{Alvarez I}.} The opinions of the Court of Appeals for the Ninth Circuit are published at \Reporter{Alvarez II} \StartPage{Alvarez II}, and 638 F.3d 666.

\section{Jurisdiction} 

This Court has jurisdiction under \index{Statute}{bb@28 U.S.C. \S1254}28 U.S.C. \S 1254(1).  The judgment of the Court of Appeals was entered on Aug. 17, 2010.  That court denied a rehearing on March 21, 2011. The petition for writ of certiorari was filed with this Court on Aug. 18, 2011, and was granted on October 17, 2011. \citetext{\See 132 S. Ct.\@ 457.}  

\section{Constitutional and Statutory Provisions Involved} 

This case implicates the Free Speech Clause of the First Amendment of the U.S. Constitution: ``Congress shall make no law \ldots abridging the freedom of speech\ldotss''

The defendant was charged with, and convicted of, a violation of the Stolen Valor Act of 2005 \citeclause{Pub. L. No.\ 109-437, \S 2(1), 120 Stat.\ 3266 (\emph{codified at} 18 U.S.C. \S 704(b) (2006))} which provides that:
\begin{quotation}
\noindent Whoever falsely represents himself or herself, verbally or in writing, to have been awarded any decoration or medal authorized by Congress for the Armed Forces of the United States, any of the service medals or badges awarded to the members of such forces, the ribbon, button, or rosette of any such badge, decoration, or medal, or any colorable imitation of such item shall be fined under this title, imprisoned not more than six months, or both.
\end{quotation}

Additionally, the Court applied the enhanced sentence imposed by Subsection C of that Act \citeclause{18 U.S.C. \S 704(c)} which provides that:
\begin{quotation}
\noindent If a decoration or medal involved in an offense under subsection (a) or (b) is a Congressional Medal of Honor, in lieu of the punishment provided in that subsection, the offender shall be fined under this title, imprisoned not more than 1 year, or both.
\end{quotation}

\section{Statement of Facts} 

\subsection{The Statute At Issue}

Congress passed the Stolen Valor Act ``to permit law enforcement officers to protect the reputation and meaning of military decorations and medals.'' \citetext{Pub.\ L.\ 109-437 \S~2(3).} It asserts that ``[f]raudulent claims surrounding the receipt of the Medal of Honor \ldots and other decorations and medals awarded by the President or the Armed Forces of the United States damage the reputation and meaning of such decorations and medals.'' \citetext{\emph{Id.} \S 2(1).}

On Veterans' Day of 2005, Senator Conrad introduced the Stolen Valor Act to ``honor the brave veterans of our Nation who have been awarded valorous medals for their service to our Nation.'' \pincite{151 Cong. Rec.}{S12689}. Such ``service and sacrifice [would] be cheapened,'' said the Senator, ``by those who wish to exploit these honors for personal gain.'' \Id. These ``individuals [] diminish the accomplishments of award recipients by using medals they have not earned \ldots [and] [t]hese fraudulent acts can often lead to the perpetration of very serious crimes.'' \Id[at S12688]. The Senate passed the bill in its present form by unanimous consent. \See \id[at S12690].

During debate on the Bill in the House of Representatives, Representative Sensenbrenner spoke first, urging the House to pass the bill out of the concern that ``the significance of these medals is being devalued by phony war heroes.'' \pincite{152 Cong. Rec.}{H8820}. By way of illustration, he stated that over 248 people falsely purport to have received the Medal of Honor. \See \id. Speaking for the minority, Representative Conyers told the House that ``this legislation represents just one of the many ways of saying thanks for a job well done.'' \Id. And Representative Kline saw the bill as ``a unique opportunity to return to our veterans and military personnel the dignity and respect taken by those who have stolen it and dishonor them.'' \Id. The House passed the bill by voice vote. \See \id[at H8823]. It was signed into law by President Bush on Dec. 20, 2006. \citetext{\See 152 Cong. Rec. H00000-14.} 

\subsection{Procedural History}

At the June 23, 2007 meeting of the Three Valley Water District Board, the newly-elected Director Xavier Alvarez, respondent here, introduced himself as a retired marine, who had retired in 2001 after 25 years of service. \See \pincite{Alvarez II}{1201}. This statement was merely a lie. He continued: ``Back in 1987, I was awarded the Congressional Medal of Honor. I got wounded many times by the same guy. I'm still around.'' \pincite{Alvarez II}{1201}. This statement was a crime. 

With the exception of ``I'm still around,'' the statement was false---Mr.\ Alvarez now concedes that he has never served in the armed forces, nor won the Medal of Honor.  \See \pincite{Alvarez II}{1200}. And so by verbalizing these twenty-three words, he violated the Stolen Valor Act, and subjected himself to a maximum of one year in prison. Mr.\ Alvarez was arrested by the FBI, and charged with two counts of violating the Stolen Valor Act. \See \id. The first count, to which he later pled guilty, alleged that: ``On or about July 23, 2007 \ldots defendant XAVIER ALVAREZ did falsely represent verbally that he had been awarded the Congressional Medal of Honor when, in truth as defendant knew, he had not received the Congressional Medal of Honor.'' \citetext{J.A. 12.} (The second count, later dropped, charged an similar offense claim allegedly committed in 2005.) \citetext{\See J.A. 12--13.}

Mr.\ Alvarez filed a motion to dismiss the indictment, on the grounds that the Stolen Valor Act is an unconstitutional abridgment of the freedom of speech guaranteed by the First Amendment. The district court denied this motion. \cite{Alvarez I}.  That court found that the statement in question ``appears to be merely a lie to impress others.'' \Id[at *3]. The district court then held that ``a false statement of fact, made knowingly and intentionally[,] \ldots [that] does not portray a political message, nor [~] deal with a matter of public debate \ldots [is] not protected by the Constitution.'' \Id.

In the face of the court's adverse ruling, Mr.\ Alvarez pled guilty to Count One of the indictment, while reserving his right to appeal the district court's denial of his motion. \See \pincite{Alvarez II}{1199}. The Court of Appeals for the Ninth Circuit reversed. \See \pincite{Alvarez II}{1218}. The panel held, over a dissent, that the Stolen Valor Act violated the First Amendment. The Government petitioned the Court of Appeals for rehearing and for rehearing \emph{en banc}. Both were denied. \citetext{\See 638 F.3d 666.}  The Government subsequently and successfully petitioned this Court for certiorari, which this Court granted on October 17, 2011. \citetext{\See 132 S. Ct. 457.}

\section{Summary of Argument}

The Stolen Valor Act is unconstitutional, because it is a content-based abridgment of speech, which violates the First Amendment. The Act criminalizes speech solely based on the content of that speech---without any requiring that the Government show that the speech caused harm. All that it requires is proof that someone ``falsely represents himself or herself \ldots to have been awarded any decoration or medal authorized by Congress for the Armed Forces of the United States.'' \citetext{18 U.S.C. \S 704(b)}. Because the First Amendment states in plain terms that 
``Congress shall make no law \ldots abridging the freedom of speech,'' this Court has applied the strictest scrutiny to any laws that criminalize speech. \See \pincite{Playboy}{813}.  That is, Congress ``must specifically identify an `actual problem' in need of solving \ldots and the curtailment of free speech must be actually \emph{necessary} to the solution.'' \citetext{\pincite{Brown}{2738} (emphasis added).} This is a challenging burden to meet. \citetext{\See \pincite{Playboy}{818} (``It is rare that a regulation restricting speech because of its content will ever be permissible'').}

And the Government has failed to meet that burden.   Not only is the Act not ``necessary'' to any compelling problem, but the problem of false honors is better solved through the First Amendment's preferred remedy of more speech, not less.  \citetext{\Cf \pincite{Johnson}{419} (``If there be time to expose through discussion the falsehood and fallacies \ldots the remedy to be applied is more speech, not enforced silence'').} The Government claims, in response, that knowingly false statements of fact are categorically exempted from the protections of the First Amendment. This is false. While Congress may restrict certain categories of false speech, such a libel and fraud \citeclause{\see \pincite{Stevens}{1584}} this Court has never recognized a general power to regulate all false statements.  \citetext{\Cf \pincite{R.A.V.}{384} (``Our cases surely do not establish the proposition that the First Amendment imposes no obstacle whatsoever to regulation of particular instances of such proscribable expression, so that the government `may regulate [them] freely'\,'').} To the contrary, the First Amendment applies even to socially undesirable speech, such as that targeted by the Act. Therefore, as the Act fails strict scrutiny, and does not fit any categorical exemption to the First Amendment, it is unconstitutional. The judgment of the Court of Appeals should be affirmed.

\section{Argument}

The First Amendment provides that ``Congress shall make no law \ldots abridging the freedom of speech.'' \textsc{U.S. Const.} amend.\ I. Such abridgments are particularly pernicious when they abridge speech based upon that speech's content, as does the Act. (\emph{See} Section~\ref{sec:strict-scrutiny}, \emph{infra}). While restrictions on certain ``well-defined and narrowly limited classes of speech \ldots have never been thought to raise any Constitutional problem,'' \pincite{Chaplinsky}{571--72}, false statements of fact are not (without more) one such class. (\emph{See} Section~\ref{sec:false-statements-not-criminalizable}, \emph{infra}). Nor is the speech punished by the Act is not among the recognized classes. (\emph{See} Section~\ref{sec:the-act-meets-no-exception}, \emph{infra}). Finally, the Act is not narrowly tailored to serve a compelling government purpose, and is therefore unconstitutional.  (\emph{See} Section~\ref{sec:less-restrictive-means-exist}, \emph{infra}).

\subsection{Content-based Restrictions On Speech, Including The Act, Face Strict Scrutiny Except In A Limited Set Of Categories}
\label{sec:strict-scrutiny}
 
%i.	About content-based regulations

``[A]s a general matter, the First Amendment provides that government has no power to restrict expression because of its message, its ideas, its subject matter, or its content.'' \pincite{Ashcroft}{573}. Accordingly, the Constitution rarely permits such restrictions. \citetext{\See \pincite{Playboy}{818} (``It is rare that a regulation restricting speech because of its content will ever be permissible'').} And ``[any] content-based speech restriction [~] can stand only if it satisfies strict scrutiny.'' \See \pincite{Playboy}{813}. That is, the restriction ``must be narrowly tailored to promote a compelling Government interest \ldotss If a less restrictive alternative would serve the Government's purpose, [Congress] must use that alternative.'' \pincite{Playboy}{813}. Such content-based restrictions are ``presumptively invalid, and the government bears the burden to rebut that presumption.'' \pincite{Playboy}{817}.

%iv.	How this is a content-based regulation

In the present case, the Act is conceded by all parties to be a content-based restriction, as it criminalized Mr.\ Alvarez's speech without any showing that his speech caused harm---or indeed, that his speech successfully deceived anyone. \See \pincite{Alvarez II}{1202}. 

%ii.	Exceptions

The First Amendment, of course, allows Congress to restrict certain activities that involve a speech component. \citetext{\See \pincite{R.A.V.}{382--83} (``From 1791 to the present, [~] our society \ldots has permitted restrictions upon the content of speech in a few limited areas'').}  But the areas in which Congress may regulate are limited to a few ``historical and traditional categories long familiar to the bar[,] includ[ing] obscenity, defamation, fraud, incitement, and speech integral to criminal conduct,'' \citetext{\pincite{Stevens}{1584} (citations and internal quotation marks omitted)}\IfLawReview{}{,} the latter of which includes (among other things): perjury \citeclause{\see \cite{Dunnigan}} criminal solicitation \citeclause{\see \cite{Williams}} threats \citeclause{\see \cite{Black}} and child pornography \citeclause{\see \pincite{Stevens}{1586} (``The market for child pornography was `intrinsically related' to the underlying abuse'') (quoting \pincite{Ferber}{759})}.  ``Fighting words,'' those subject to provoking an immediate breach of the peace, can also be subject to government sanctions as a form of incitement. \See \cite{Chaplinsky}.

%iii.	Congress cannot expand this list

These categories are fixed, and derive from the understandings of freedom enshrined in the First Amendment at its ratification. \citetext{\Cf \pincite{Meyer}{420--21} (``[E]very person must be his own watchman for truth, because the forefathers did not trust any government to separate the true from the false for us.'').}  Congress cannot expand them to novel areas such as the false claims criminalized by the Act. \citetext{\See \pincite{Brown}{2734} (``new categories of unprotected speech may not be added to the list by a legislature that concludes certain speech is too harmful to be tolerated'').} Moreover, the contours of these categories are clear, such that ``no process of case-by-case adjudication is required'' to determine whether speech is contained in a given category. \See \pincite{Ferber}{763--64}.

%Mention the case-by-case adjudicatio n

These categories do \emph{not} derive from an assessment of the value of such speech, or the lack thereof. \citetext{\See \pincite{Stevens}{1585} (``free speech does not extend only to categories of speech that survive an ad hoc balancing of relative social costs and benefits'').}  Purportedly worthless speech, therefore, is within the scope of the First Amendment. \citetext{\Seeeg \pincite{Brown}{} (``Crudely violent video games, tawdry TV shows, and cheap novels and magazines are no less forms of speech than The Divine Comedy.''); \pincite{Stevens}{} (``Most of what we say to one another lacks `religious, political, scientific, educational, journalistic, historical, or artistic value' (let alone serious value), but it is still sheltered from government regulation.''); \pincite{Winters}{510} (``Though we can see nothing of any possible value to society in these magazines, they are as much entitled to the protection of free speech as the best of literature.'').}  And this Court has refused to recognize exceptions for purportedly harmful or hurtful speech. \citetext{\Seeeg \cite{Brown} (depictions of violence); \cite{Stevens} (depictions of animal cruelty); \cite{Snyder} (protests at funerals); \cite{Hustler} (crude innuendo inflicting emotional distress); \cite{Johnson} (flag burning).} And even commercial speech, often subject to greater regulation, is not entirely outside the scope of the free speech clause. \See \pincite{Va. State Bd. of Pharmacy}{770}. Thus, unless the Government can prove that false claims fall into one of the ``historical and traditional categories'' subject to regulation, the Act is subject to the same strict scrutiny as any other content-based speech restriction.

\subsection{Congress May Not Criminalize Statements For The Mere Fact That They Are False}
\label{sec:false-statements-not-criminalizable}

False claims about valorous medals, however, are not a ``historically and traditionally'' recognized category of regulated speech. So the Government instead claims that Congress may criminalize \emph{all} knowingly false statements of fact ``unless immunity has been carved out \ldots to protect speech that matters.'' \See \pincite{Alvarez II}{1203} (quoting petitioner's brief). This claim is false. Falsity \emph{alone} is not enough to remove a speaker or his speech from the First Amendment's protection. \citetext{\Cf \pincite{New York Times}{271} (``Authoritative interpretations of the First Amendment guarantees have consistently refused to recognize an exception for any test of truth'').}

%Positive case should go here, perhaps the analogy to petition clause 

The Government and the dissents below, however, believe that this Court has endorsed such a broad claim with its occasionally repeated comment that ``the erroneous statement of fact is not worthy of constitutional protection.'' \citetext{\pincite{Alvarez II}{1218} (Bybee, J., dissenting) (quoting \pincite{Gertz}{340}). \Seealso \pincite{BE & K Construction}{531} (``false statements [are] unprotected for their own sake''); \pincite{Hustler}{52} (``False statements of fact are particularly valueless''); \pincite{Keeton}{776} (false statements of fact have ``no constitutional value''); \pincite{Bill Johnson's}{743} (``false statements are not immunized by the First Amendment''); \pincite{Herbert}{161} (''Spreading false information in and of itself carries no First Amendment credentials.''); \pincite{Garrison}{75} (``the knowingly false statement \ldots do[es] not enjoy constitutional protection.'').}  The government and the dissenters interpret these comments as holding that, because Congress may regulate false statements in some cases, that Congress may regulate false statements in \emph{any} case, as it sees fit.

They are wrong. While this Court has observed that certain ``categories of speech [~] normally receive reduced First Amendment protection, or no First Amendment protection at all, [the Court] [has] never held that the government may regulate speech within those categories in any way that it wishes.'' \pincite{Lorillard}{553}. Regarding this Court's prior statements that some categories of speech are ``unprotected,'' this Court clarified that ``[s]uch statements must be taken in context, [~] and are no more literally true than is the occasionally repeated shorthand characterizing obscenity `as not being speech at all.'\,'' \pincite{R.A.V.}{383}. In contrast to the sweeping meaning that the Government would assign to them, ``[w]hat they mean is that these areas of speech can \ldots be regulated because of their constitutionally proscribable content (obscenity, defamation, etc.)---not that they are categories of speech entirely invisible to the Constitution.'' \citetext{\Id. (citation and internal quotation marks omitted).} And so this Court accordingly ``[has] applied heightened scrutiny to laws [regarding] \ldots speech not protected by the First Amendment.'' \pincite{Carrigan}{2349} 

Moreover, the \emph{R.A.V.} Court explicitly rejected the Government's and the Concurrence's view in that case that Congress has free reign over ``unprotected'' speech. \citetext{\See \pincite{R.A.V.}{384} (``Our cases surely do not establish the proposition that the First Amendment imposes no obstacle whatsoever to regulation of particular instances of such proscribable expression, so that the government `may regulate [them] freely'\,'') (quoting \pincite{R.A.V.}{400} (White, J., concurring in judgment)).} Rather, ``[e]ven when speech falls into a category of reduced constitutional protection, the government may not engage in content discrimination for reasons unrelated to those characteristics of the speech that place it within the category.'' \pincite{Lorillard}{576}. Therefore, while legislatures ``[may] choose prohibit only that obscenity which is the most patently offensive \emph{in its prurience} \ldots it may not prohibit \ldots only that obscenity which includes offensive \emph{political} messages.'' \citetext{\pincite{R.A.V.}{388} (emphases in original).} Nor may it ``criminalize only those threats against the President that mention his policy on aid to inner cities.'' \pincite{R.A.V.}{388}. Nor may it ``prohibit only that commercial advertising that depicts men in a demeaning fashion.'' \pincite{R.A.V.}{389}. It does not follow that the First Amendment would restrict Congress's ability to regulate certain threats, obscenities, advertisements, and libels, only to allow Congress arbitrary powers to criminalize (non-libelous) false statements. 

Nor do the cases cited by the Government or the dissenters support such a claim.  All of them correspond to a traditional category of restricted speech, and do not rely on any general power to ban false statements of fact. Most are libel or defamation. \citetext{\See \cite{Gertz} (allowing libel claim to proceed); \cite{Keeton} (holding distribution in a state to create minimum contacts for jurisdiction over a libel suit); \cite{Garrison} (holding unconstitutional a criminal defamation statue with no \emph{mens rea} requirement); \cite{Herbert} (allowing libel plaintiff to utilize discovery to prove knowledge).}  By contrast, in \emph{Hustler} \citeclause{\cite[n]{Hustler}} notwithstanding the Court's characterization of false statements as ``valueless,'' \citetext{\IfLawReview{\Id.}{\id.} at 52\IfLawReview{.}{}} the Court overturned on First Amendment grounds a claim of intentional infliction of emotion distress, where defendant knowingly falsely represented that plaintiff lost his virginity through incestuous sex with his mother in an outhouse. 

Two cases involved allegedly baseless lawsuits. The Court in \cite{Bill Johnson's}, considered (and then restricted) the power of the National Labor Relations Board to enjoin state lawsuits between employers and employees; this Court mentioned ``false statements'' merely by way of analogy to baseless lawsuits. \citetext{\See \pincite[n]{Bill Johnson's}{743} (``Just as false statements are not immunized by the First Amendment right to freedom of speech, baseless litigation is not immunized by the First Amendment right to petition.'') (citations omitted).} And in \pincite{BE & K Construction}{531}, this Court clarified that ``[w]hile this analogy is helpful, it does not suggest that the class of baseless litigation is completely unprotected: At most, it indicates such litigation should be unprotected `just as' false statements are.'' \pincite{BE & K Construction}{531}. Therefore, the Court has ``ha[s] never held that the entire class of objectively baseless litigation may be enjoined or declared unlawful even though such suits may advance no First Amendment interests of their own.'' \Id[at 531]. 

Given that false statements are protected (or unprotected) ``just as'' baseless lawsuits (or libels, threats, and obscenities), neither may the entire class of objectively false statements be declared unlawful.  Instead, false statements may be limited in particular situations---precisely the ``historical and traditional'' categories, such as fraud and libel. \citetext{\Cf  \pincite{Stevens}{1584} (listing unprotected categories).}  Therefore, the Government must show that false claims about medals falls into one of the existing categories of criminalizable speech, not that it may be criminalized by the mere fact of its falsity. 

\subsection{The Act Does Not Meet Any Recognized Exception to Strict Scrutiny}
\label{sec:the-act-meets-no-exception}

The Stolen Valor Act does not, however, correspond to any of the existing categories that Congress may regulate. First, it does not constitute fraud. \citetext{\See \pincite{Madigan}{620} (``False statement alone does not subject a [defendant] to fraud liability.'').} Rather, ``in a properly tailored fraud action \ldots the complainant must demonstrate that the defendant made the [false] representation with the intent to mislead the listener, and succeeded in doing so.'' \pincite{Madigan}{620}. Moreover, ``the words `to defraud' \ldots have the common understanding of wronging one in his property rights by dishonest methods or schemes, and usually signify the deprivation of something of value.'' \pincite{Carpenter}{27}. 

The Act does not meet the definition of a fraud statute. Notwithstanding references to ``[f]raudulent claims,'' Pub. L.\ 109-437 \S 2(3), and ``fraudulent acts,'' \citetext{\pincite{151 Cong. Rec.}{at S12688} (Statement of Sen. Conrad)\PeriodOrComma} ``[s]imply labeling an action one for `fraud,' of course, will not carry the day.'' \pincite{Madigan}{620}. The Act does not require that another person is involved in any way, but rather punishes ``Whoever falsely represents himself or herself \ldots to have been awarded any decoration or medal \ldotss'' 18 U.S.C. \S 704(b). That is, it does not require even an ``intent to mislead,'' much less that the defendant ``succeeded in doing so.'' Nor does it require any showing that the speaker ``wrong[ed] one in his property rights,'' or caused ``the deprivation of something of value.'' In short, the Act bears no resemblance to a fraud statute.  Indeed, the record shows that ``if anything, Alvarez has no credibility whatsoever and that no one detrimentally relied on his false statement.'' \pincite{Alvarez II}{1212}. He did not commit fraud. 

Its drafters may have intended the statute to preempt actual fraud before it occurs. \citetext{\Cf \pincite{151 Cong. Rec.}{at S12688} (``[t]hese fraudulent acts can often lead to the perpetration of very serious crimes.'').} But ``[b]road prophylactic rules in the area of free expression are suspect.'' \pincite{NAACP}{438}. Thus the First Amendment prohibits regulations purported to combat fraud where ``the interest is not [~] weighty,'' and where ``the means chosen to accomplish it are unduly burdensome and not narrowly tailored.'' \citetext{\pincite{Riley}{798}. \Accord \cite{Munson}; \cite{Schaumburg}.} Even under the reasonable assumption that preventing fraud is a ``weighty'' interest, the Act is not tailored towards those acts that would potentially lead to fraud. It is thus invalid as a prophylactic measure. 

Nor is the Act a valid defamation statute, as has been suggested. \citetext{\Seeeg Pub. L.\ 109-437 \S 2(3) (stating the Act's purpose to ``to protect the reputation \ldots of military decorations'');  \id. \S 2(1) (``[f]raudulent claims \ldots damage the [medals'] reputation'').} But defamation requires not only falsity, but harm. \citetext{\See \pincite{Dun & Bradstreet}{771} (``misstatements of fact that seriously harm the reputation of another'') (citing \pincite{Restatement of Torts}{\S 559}).} A showing of harm is absent here. Nor does the Act require one. Nor is there a basis to infer that harm occured. As the Court of Appeals observed, ``there is no readily apparent reason for assuming, without specific proof, that the reputation and meaning of military decorations is harmed every time someone lies about having received one.'' \pincite{Alvarez II}{1210}. Rather, such liars presumably believe that ``their being perceived as recipients of such honors brings them acclaim, suggesting that generally the integrity and reputation of such honors remain unimpaired.'' \Id.

And as a general matter, only natural persons---not governments---have a right against defamation. \citetext{\See \pincite{New York Times}{291} (``[N]o court of last resort in this country has ever held, or even suggested, that prosecutions for libel on government have any place in the American system of jurisprudence.'').} Thus, while our flag is a symbol of our Nation worthy of honorable treatment, ``[t]o say that the government has an interest in encouraging proper treatment of the flag, however, is not to say that it may criminally punish a person for burning a flag as a means of political protest.'' \See \pincite{Johnson}{418--20}. 

%\subsection{The Stolen Valor Act Fails Strict Scrutiny}


\subsection{Congress May Honor Our Service Members With Less Restrictive and More Narrowly-Tailored Measures}
\label{sec:less-restrictive-means-exist}

The Stolen Valor Act, as a content-based speech regulation that does not fit a category of permissible regulations, is subject to strict scrutiny.  It ``must be narrowly tailored to promote a compelling Government interest \ldotss If a less restrictive alternative would serve the Government's purpose, [Congress] must use that alternative.'' \pincite{Playboy}{813}. The Act does not meet this test.

Congress's purposes in passing the Stolen Valor Act were, in the words of its congressional proponents, to ``honor the brave veterans of our nation,'' 
\citetext{\pincite{151 Cong. Rec.}{at S12689} (Sen. Conrad)\PeriodOrComma} ``to say[] thanks for a job well done,'' \citetext{\pincite{152 Cong. Rec.}{at H2280} (Rep. Conyers)\PeriodOrComma} and ``to return to our veterans and military personnel the[ir] dignity and respect\PeriodOrComma'' \citetext{\Id[at H2281]\ (Rep. Kline).} These goals are certainly noble, but Congress could achieve these goals by less restrictive means. 

Most importantly, any harm done by Mr.\ Alvarez and his ilk could be remedied less restrictively through \emph{more} speech, not less. \citetext{\Cf \pincite{Johnson}{419} (``If there be time to expose through discussion the falsehood and fallacies \ldots the remedy to be applied is more speech, not enforced silence'').} That is, such fakes are best handled by exposing them for what they are. And ``[i]f a less restrictive alternative would serve the Government's purpose, [Congress] must use that alternative.'' \pincite{Playboy}{813}. Indeed, Alvarez was exposed quickly (if ever believed at all) by the press and the public, before the FBI investigation, with one article describing him as an ``idiot,'' and another as a ``jerk.'' \See \pincite{Alvarez II}{1211}. The First Amendment's preferred remedy of ``more speech'' seems to have achieved the Act's purpose to thwart Alvarez's ``wish to exploit th[is] honor for personal gain.'' \Cf \pincite{151 Cong. Rec.}{at S12689}. Congress has the power to take other measures to achieve it's worthy end; thus the Act fails strict scrutiny, and is unconstitutional. 

Finally, it is speculative at best to conclude that criminally-punishing lies is the only way to ensure the reputation of such medals. Rather, ''it seems just as likely that the reputation and meaning of such medals is wholly unaffected by those who lie about having received them. The greatest damage done seems to be to the reputations of the liars themselves.'' \pincite[s]{Alvarez II}{1217}. While society would, no doubt, be better off with fewer liars, ``[t]he First Amendment itself reflects a judgment by the American people that the benefits of its restrictions on the Government outweigh the costs.'' \pincite{Stevens}{1585}. The Stolen Valor Act is incompatible with the First Amendment, and was properly held unconstitutional by the Court of Appeals.

\section{Conclusion}

\pincite{Hustler}{1} \pincite{Alvarez II}{1} \cite{Stevens}

The judgment of the Court of Appeals should be affirmed.
\vspace{.3in}

\makeatletter
\mbox{\hspace{3.5in} \vbox{\raggedright Respectfully Submitted, \\ \@author \\ February 27, 2012}}
\makeatother


%\end{ignore}
\end{document}
